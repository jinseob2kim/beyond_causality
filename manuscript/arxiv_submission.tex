% Options for packages loaded elsewhere
\PassOptionsToPackage{unicode}{hyperref}
\PassOptionsToPackage{hyphens}{url}
\PassOptionsToPackage{dvipsnames,svgnames,x11names}{xcolor}
%
\documentclass[
  11pt,
]{article}

\usepackage{amsmath,amssymb}
\usepackage{setspace}
\usepackage{iftex}
\ifPDFTeX
  \usepackage[T1]{fontenc}
  \usepackage[utf8]{inputenc}
  \usepackage{textcomp} % provide euro and other symbols
\else % if luatex or xetex
  \usepackage{unicode-math}
  \defaultfontfeatures{Scale=MatchLowercase}
  \defaultfontfeatures[\rmfamily]{Ligatures=TeX,Scale=1}
\fi
\usepackage{lmodern}
\ifPDFTeX\else  
    % xetex/luatex font selection
\fi
% Use upquote if available, for straight quotes in verbatim environments
\IfFileExists{upquote.sty}{\usepackage{upquote}}{}
\IfFileExists{microtype.sty}{% use microtype if available
  \usepackage[]{microtype}
  \UseMicrotypeSet[protrusion]{basicmath} % disable protrusion for tt fonts
}{}
\makeatletter
\@ifundefined{KOMAClassName}{% if non-KOMA class
  \IfFileExists{parskip.sty}{%
    \usepackage{parskip}
  }{% else
    \setlength{\parindent}{0pt}
    \setlength{\parskip}{6pt plus 2pt minus 1pt}}
}{% if KOMA class
  \KOMAoptions{parskip=half}}
\makeatother
\usepackage{xcolor}
\usepackage[margin=1in]{geometry}
\setlength{\emergencystretch}{3em} % prevent overfull lines
\setcounter{secnumdepth}{-\maxdimen} % remove section numbering
% Make \paragraph and \subparagraph free-standing
\makeatletter
\ifx\paragraph\undefined\else
  \let\oldparagraph\paragraph
  \renewcommand{\paragraph}{
    \@ifstar
      \xxxParagraphStar
      \xxxParagraphNoStar
  }
  \newcommand{\xxxParagraphStar}[1]{\oldparagraph*{#1}\mbox{}}
  \newcommand{\xxxParagraphNoStar}[1]{\oldparagraph{#1}\mbox{}}
\fi
\ifx\subparagraph\undefined\else
  \let\oldsubparagraph\subparagraph
  \renewcommand{\subparagraph}{
    \@ifstar
      \xxxSubParagraphStar
      \xxxSubParagraphNoStar
  }
  \newcommand{\xxxSubParagraphStar}[1]{\oldsubparagraph*{#1}\mbox{}}
  \newcommand{\xxxSubParagraphNoStar}[1]{\oldsubparagraph{#1}\mbox{}}
\fi
\makeatother


\providecommand{\tightlist}{%
  \setlength{\itemsep}{0pt}\setlength{\parskip}{0pt}}\usepackage{longtable,booktabs,array}
\usepackage{calc} % for calculating minipage widths
% Correct order of tables after \paragraph or \subparagraph
\usepackage{etoolbox}
\makeatletter
\patchcmd\longtable{\par}{\if@noskipsec\mbox{}\fi\par}{}{}
\makeatother
% Allow footnotes in longtable head/foot
\IfFileExists{footnotehyper.sty}{\usepackage{footnotehyper}}{\usepackage{footnote}}
\makesavenoteenv{longtable}
\usepackage{graphicx}
\makeatletter
\newsavebox\pandoc@box
\newcommand*\pandocbounded[1]{% scales image to fit in text height/width
  \sbox\pandoc@box{#1}%
  \Gscale@div\@tempa{\textheight}{\dimexpr\ht\pandoc@box+\dp\pandoc@box\relax}%
  \Gscale@div\@tempb{\linewidth}{\wd\pandoc@box}%
  \ifdim\@tempb\p@<\@tempa\p@\let\@tempa\@tempb\fi% select the smaller of both
  \ifdim\@tempa\p@<\p@\scalebox{\@tempa}{\usebox\pandoc@box}%
  \else\usebox{\pandoc@box}%
  \fi%
}
% Set default figure placement to htbp
\def\fps@figure{htbp}
\makeatother
% definitions for citeproc citations
\NewDocumentCommand\citeproctext{}{}
\NewDocumentCommand\citeproc{mm}{%
  \begingroup\def\citeproctext{#2}\cite{#1}\endgroup}
\makeatletter
 % allow citations to break across lines
 \let\@cite@ofmt\@firstofone
 % avoid brackets around text for \cite:
 \def\@biblabel#1{}
 \def\@cite#1#2{{#1\if@tempswa , #2\fi}}
\makeatother
\newlength{\cslhangindent}
\setlength{\cslhangindent}{1.5em}
\newlength{\csllabelwidth}
\setlength{\csllabelwidth}{3em}
\newenvironment{CSLReferences}[2] % #1 hanging-indent, #2 entry-spacing
 {\begin{list}{}{%
  \setlength{\itemindent}{0pt}
  \setlength{\leftmargin}{0pt}
  \setlength{\parsep}{0pt}
  % turn on hanging indent if param 1 is 1
  \ifodd #1
   \setlength{\leftmargin}{\cslhangindent}
   \setlength{\itemindent}{-1\cslhangindent}
  \fi
  % set entry spacing
  \setlength{\itemsep}{#2\baselineskip}}}
 {\end{list}}
\usepackage{calc}
\newcommand{\CSLBlock}[1]{\hfill\break\parbox[t]{\linewidth}{\strut\ignorespaces#1\strut}}
\newcommand{\CSLLeftMargin}[1]{\parbox[t]{\csllabelwidth}{\strut#1\strut}}
\newcommand{\CSLRightInline}[1]{\parbox[t]{\linewidth - \csllabelwidth}{\strut#1\strut}}
\newcommand{\CSLIndent}[1]{\hspace{\cslhangindent}#1}

\usepackage{amsmath}
\usepackage{amssymb}
\makeatletter
\@ifpackageloaded{caption}{}{\usepackage{caption}}
\AtBeginDocument{%
\ifdefined\contentsname
  \renewcommand*\contentsname{Table of contents}
\else
  \newcommand\contentsname{Table of contents}
\fi
\ifdefined\listfigurename
  \renewcommand*\listfigurename{List of Figures}
\else
  \newcommand\listfigurename{List of Figures}
\fi
\ifdefined\listtablename
  \renewcommand*\listtablename{List of Tables}
\else
  \newcommand\listtablename{List of Tables}
\fi
\ifdefined\figurename
  \renewcommand*\figurename{Figure}
\else
  \newcommand\figurename{Figure}
\fi
\ifdefined\tablename
  \renewcommand*\tablename{Table}
\else
  \newcommand\tablename{Table}
\fi
}
\@ifpackageloaded{float}{}{\usepackage{float}}
\floatstyle{ruled}
\@ifundefined{c@chapter}{\newfloat{codelisting}{h}{lop}}{\newfloat{codelisting}{h}{lop}[chapter]}
\floatname{codelisting}{Listing}
\newcommand*\listoflistings{\listof{codelisting}{List of Listings}}
\makeatother
\makeatletter
\makeatother
\makeatletter
\@ifpackageloaded{caption}{}{\usepackage{caption}}
\@ifpackageloaded{subcaption}{}{\usepackage{subcaption}}
\makeatother

\usepackage{bookmark}

\IfFileExists{xurl.sty}{\usepackage{xurl}}{} % add URL line breaks if available
\urlstyle{same} % disable monospaced font for URLs
\hypersetup{
  pdftitle={Epidynamix: From Force to Field in Real-World Epidemiology},
  pdfauthor={Jinseob Kim},
  pdfkeywords={causal inference, real-world data, positivity
violation, treatment effect heterogeneity, risk potential field},
  colorlinks=true,
  linkcolor={blue},
  filecolor={Maroon},
  citecolor={Blue},
  urlcolor={Blue},
  pdfcreator={LaTeX via pandoc}}


\title{Epidynamix: From Force to Field in Real-World Epidemiology}
\author{Jinseob Kim}
\date{}

\begin{document}
\maketitle
\begin{abstract}
Standard causal inference methods estimate average treatment effects
(ATE) under positivity and no unmeasured confounding assumptions. In
real-world data (RWD), structural positivity violations arise from
clinical guidelines and contraindications, while treatment effects are
often heterogeneous across patient states. We propose Epidynamix, a
framework that models health states as points within a Risk Potential
Field \(\Phi(s) = -\log\lambda(s)\), where treatment effects become
directional gradients and positivity violations define structural
boundaries. Through simulation, we show that while ATE-based methods
(Cox regression, IPTW) collapse effect heterogeneity into scalar
summaries, the field-based approach preserves clinically meaningful
variation (\textasciitilde4-fold across state space). This geometric
perspective offers a complementary lens for RWD analysis, particularly
when the standard causal question is structurally unanswerable.
\end{abstract}


\setstretch{1.5}
\section{Introduction}\label{introduction}

For decades, causal inference in epidemiology has operated under a
``Newtonian'' abstraction: treatment \(A\) exerts a force on outcome
\(Y\), producing an Average Treatment Effect (ATE) that we estimate as
if it were a universal constant.

This framework assumes: (1) interventions are external forces applied to
passive subjects; (2) positivity violations are statistical nuisances to
be trimmed; and (3) a single scalar (ATE or Hazard Ratio) captures the
truth of treatment impact.

\textbf{Why now?} The explosion of Real-World Data (RWD)---electronic
health records, claims databases, registries---has exposed the
limitations of this paradigm. Unlike curated trial populations, RWD
reflects the full complexity of clinical practice: treatment decisions
are constrained by guidelines, contraindications create structural zeros
in the propensity score, and effects vary systematically across patient
states. We increasingly face situations where:

\begin{itemize}
\tightlist
\item
  Treatment is mandated for high-risk patients (structural positivity
  violation)
\item
  The ``same'' intervention varies by context, timing, and patient state
\item
  Treatment decisions are entangled with prognosis (confounding by
  indication)
\end{itemize}

Traditional methods respond by restricting analysis to ``overlap
regions'' or trimming extreme weights---effectively discarding the very
patients for whom treatment decisions matter most (Petersen et al.
2012).

We propose an alternative: \textbf{a map, not a number.}

\section{The Geometric Turn: From Force to
Field}\label{the-geometric-turn-from-force-to-field}

We propose a fundamental shift: \textbf{Causal effects are not primitive
forces; they are movements within a structured Field.} We call this
framework Epidynamix.

\subsection{The Risk Potential Field}\label{the-risk-potential-field}

We define a scalar field over the state space \(\mathcal{S}\) of all
possible patient conditions, called the Risk Potential:

\[
\Phi(s) = -\log(\lambda(s))
\]

where \(\lambda(s)\) is the instantaneous hazard at state \(s\). In this
field, ``safety'' is high ground (peaks), and ``danger'' is low ground
(valleys).

\subsection{Intervention as
State-Transition}\label{intervention-as-state-transition}

An intervention is not a force applied to a static object. It is an
operator \(\mathcal{I}_a\) that moves a patient from one coordinate in
state space to another. For binary treatment, the effect is:

\[
\delta_A \Phi(x) = \Phi(x, A=1) - \Phi(x, A=0)
\]

The ATE is simply the population average of this local effect:

\[
\text{ATE} = \mathbb{E}_X[\delta_A \Phi(X)]
\]

This average is information-preserving only when
\(\text{Var}_X[\delta_A \Phi] \approx 0\). When effects vary across
patient states, the ATE collapses a rich distribution into a single
number.

\section{Positivity Violations as Structural
Boundaries}\label{positivity-violations-as-structural-boundaries}

In standard causal inference, if a subgroup always receives treatment
(\(P(A=1|X) = 1\)), we call it a positivity violation---a statistical
problem to fix.

In the Epidynamix framework, this is a \textbf{Structural Cliff}: a
boundary where the untreated state is forbidden by ethics, guidelines,
or biological reality. Rather than treating these as estimation
failures, we map them as the defining edges of the clinical landscape.

\section{Simulation Study}\label{simulation-study}

To demonstrate the Field approach, we simulated a clinical scenario with
a structural boundary and heterogeneous effects.

\textbf{Setup:}

\begin{itemize}
\tightlist
\item
  \(X_1\): Systolic blood pressure (100--200 mmHg)
\item
  \(X_2\): Inflammatory marker (0--10 mg/L)
\item
  \(A\): Antihypertensive treatment (binary)
\item
  Structural constraint: If \(X_1 > 160\), then \(A = 1\) (mandatory per
  guideline)
\end{itemize}

Treatment benefit was designed to increase with \(X_1\).

\textbf{Results (N = 3,000):}

\begin{longtable}[]{@{}
  >{\raggedright\arraybackslash}p{(\linewidth - 4\tabcolsep) * \real{0.2500}}
  >{\raggedright\arraybackslash}p{(\linewidth - 4\tabcolsep) * \real{0.2500}}
  >{\raggedright\arraybackslash}p{(\linewidth - 4\tabcolsep) * \real{0.5000}}@{}}
\toprule\noalign{}
\begin{minipage}[b]{\linewidth}\raggedright
Method
\end{minipage} & \begin{minipage}[b]{\linewidth}\raggedright
Output
\end{minipage} & \begin{minipage}[b]{\linewidth}\raggedright
Interpretation
\end{minipage} \\
\midrule\noalign{}
\endhead
\bottomrule\noalign{}
\endlastfoot
Cox HR & 0.55 & Treatment reduces hazard by 45\% \\
IPTW ATE & 0.20 & Treatment increases 1-year survival by 20 percentage
points \\
Field \(\delta_A \Phi\) & 0.42--1.66 & Effect varies 4-fold across state
space \\
\end{longtable}

Traditional methods produce single numbers. The Field approach reveals
that treatment benefit varies \textasciitilde4-fold across the state
space, with the largest effects in high-BP patients---precisely those
for whom the guideline mandates treatment (Cole and Hernán 2008). The
region \(X_1 > 160\) is not a failure; it is a Structural Cliff
revealing clinical logic.

\begin{figure}

\centering{

\includegraphics[width=0.9\linewidth,height=\textheight,keepaspectratio]{fig_field_approach.png}

}

\caption{\label{fig-field}Treatment effect field showing heterogeneity
and structural boundary at \(X_1 > 160\). Gray region indicates
positivity violation where comparison is structurally impossible.}

\end{figure}%

\section{Relation to Existing
Approaches}\label{relation-to-existing-approaches}

\subsection{Differentiation from CATE}\label{differentiation-from-cate}

Conditional Average Treatment Effect (CATE) estimation---via causal
forests (Wager and Athey 2018), meta-learners, or Bayesian
methods---also addresses heterogeneity. The difference lies in the
question asked:

\begin{longtable}[]{@{}
  >{\raggedright\arraybackslash}p{(\linewidth - 4\tabcolsep) * \real{0.3077}}
  >{\raggedright\arraybackslash}p{(\linewidth - 4\tabcolsep) * \real{0.2308}}
  >{\raggedright\arraybackslash}p{(\linewidth - 4\tabcolsep) * \real{0.4615}}@{}}
\toprule\noalign{}
\begin{minipage}[b]{\linewidth}\raggedright
Aspect
\end{minipage} & \begin{minipage}[b]{\linewidth}\raggedright
CATE
\end{minipage} & \begin{minipage}[b]{\linewidth}\raggedright
Epidynamix
\end{minipage} \\
\midrule\noalign{}
\endhead
\bottomrule\noalign{}
\endlastfoot
Question & What is the effect for subgroup \(X=x\)? & What is the
structure of the risk landscape? \\
Positivity & Estimation problem & Structural information \\
Counterfactuals & Required & Not required in boundary regions \\
Output & \(\tau(x)\) & \(\Phi(s)\), gradients, boundaries \\
\end{longtable}

CATE asks ``what is the effect?''---conditional on covariates.
Epidynamix asks ``what does the landscape look like, and where can we
not go?'' Crucially, CATE treats positivity violations as missing data;
Epidynamix reports boundaries explicitly as part of the result.

\subsection{Connection to Structural Nested
Models}\label{connection-to-structural-nested-models}

The local effect \(\delta_A \Phi(x)\) shares conceptual ground with the
``blip function'' in Structural Nested Mean Models (SNMM) (Robins 1994),
which also models treatment effects as functions of patient state. The
key difference is interpretive: SNMM aims to estimate causal effects
under sequential exchangeability, while Epidynamix treats the effect
surface as a geometric object to be mapped, with boundaries as
informative features rather than obstacles.

\subsection{Novelty Claim}\label{novelty-claim}

We do not claim to invent new mathematics. Potential landscapes exist in
physics and systems biology (e.g., Waddington's epigenetic landscape
(Waddington 1957)). Our contribution is applying this lens to clinical
RWD, where positivity violations are common, effects are entangled with
state, and the average-effect question may be structurally unanswerable.

This is a reinterpretation, not an invention---a new language for an old
problem.

\section{Discussion}\label{discussion}

The Epidynamix framework offers a complementary perspective to standard
causal inference (Rubin 1974; Hernán and Robins 2020). It does not
replace existing methods; rather, it clarifies their domain of validity.

\textbf{When ATE works:} If the risk field is smooth, low-curvature, and
well-connected (positivity holds everywhere), the ATE is an accurate
summary. Traditional methods excel here.

\textbf{When ATE fails:} If effects are highly heterogeneous, or if
structural constraints create disconnected regions (Petersen et al.
2012), the ATE may be misleading or undefined. The field approach
provides richer output---a map rather than a number.

\textbf{Practical implications:}

\begin{enumerate}
\def\labelenumi{\arabic{enumi}.}
\tightlist
\item
  Report effect heterogeneity alongside average effects
\item
  Visualize the treatment effect landscape when possible
\item
  Treat positivity violations as findings, not errors
\item
  Consider whether the causal question is answerable before estimating
\end{enumerate}

\textbf{Limitations:} The framework is conceptual; practical estimation
of \(\Phi(s)\) requires flexible models and careful validation.
Extension to time-varying treatments and high-dimensional states remains
an open challenge.

\section{Conclusion}\label{conclusion}

Newtonian mechanics is not wrong---it is flat-space physics. Causal
inference is not false---it is a low-curvature approximation.

Real-world epidemiology, with its structural constraints and
heterogeneous effects, may require a geometric view. The Epidynamix
framework offers one such lens: treating causal effects not as forces,
but as directional movements within a structured risk field.

When the standard causal question cannot be answered, perhaps we should
ask a different question.

\section{References}\label{references}

\phantomsection\label{refs}
\begin{CSLReferences}{1}{0}
\bibitem[\citeproctext]{ref-cole2008}
Cole, Stephen R, and Miguel A Hernán. 2008. {``Constructing Inverse
Probability Weights for Marginal Structural Models.''} \emph{American
Journal of Epidemiology} 168 (6): 656--64.

\bibitem[\citeproctext]{ref-hernan2020}
Hernán, Miguel A, and James M Robins. 2020. \emph{Causal Inference: What
If}. Boca Raton: Chapman \& Hall/CRC.

\bibitem[\citeproctext]{ref-petersen2012}
Petersen, Maya L, Kristin E Porter, Susan Comfort, Rachel
Morello-Frosch, and Rose Ray. 2012. {``Diagnosing and Responding to
Violations in the Positivity Assumption.''} \emph{Statistical Methods in
Medical Research} 21 (1): 31--54.

\bibitem[\citeproctext]{ref-robins1994}
Robins, James M. 1994. {``Correcting for Non-Compliance in Randomized
Trials Using Structural Nested Mean Models.''} \emph{Communications in
Statistics-Theory and Methods} 23 (8): 2379--2412.

\bibitem[\citeproctext]{ref-rubin1974}
Rubin, Donald B. 1974. {``Estimating Causal Effects of Treatments in
Randomized and Nonrandomized Studies.''} \emph{Journal of Educational
Psychology} 66 (5): 688.

\bibitem[\citeproctext]{ref-waddington1957}
Waddington, Conrad Hal. 1957. \emph{The Strategy of the Genes: A
Discussion of Some Aspects of Theoretical Biology}. London: George Allen
\& Unwin.

\bibitem[\citeproctext]{ref-wager2018}
Wager, Stefan, and Susan Athey. 2018. {``Estimation and Inference of
Heterogeneous Treatment Effects Using Causal Forests.''} \emph{Journal
of the American Statistical Association} 113 (523): 1228--42.

\end{CSLReferences}




\end{document}
