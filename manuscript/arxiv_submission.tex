% Options for packages loaded elsewhere
\PassOptionsToPackage{unicode}{hyperref}
\PassOptionsToPackage{hyphens}{url}
\PassOptionsToPackage{dvipsnames,svgnames,x11names}{xcolor}
%
\documentclass[
  11pt,
]{article}

\usepackage{amsmath,amssymb}
\usepackage{setspace}
\usepackage{iftex}
\ifPDFTeX
  \usepackage[T1]{fontenc}
  \usepackage[utf8]{inputenc}
  \usepackage{textcomp} % provide euro and other symbols
\else % if luatex or xetex
  \usepackage{unicode-math}
  \defaultfontfeatures{Scale=MatchLowercase}
  \defaultfontfeatures[\rmfamily]{Ligatures=TeX,Scale=1}
\fi
\usepackage{lmodern}
\ifPDFTeX\else  
    % xetex/luatex font selection
\fi
% Use upquote if available, for straight quotes in verbatim environments
\IfFileExists{upquote.sty}{\usepackage{upquote}}{}
\IfFileExists{microtype.sty}{% use microtype if available
  \usepackage[]{microtype}
  \UseMicrotypeSet[protrusion]{basicmath} % disable protrusion for tt fonts
}{}
\makeatletter
\@ifundefined{KOMAClassName}{% if non-KOMA class
  \IfFileExists{parskip.sty}{%
    \usepackage{parskip}
  }{% else
    \setlength{\parindent}{0pt}
    \setlength{\parskip}{6pt plus 2pt minus 1pt}}
}{% if KOMA class
  \KOMAoptions{parskip=half}}
\makeatother
\usepackage{xcolor}
\usepackage[margin=1in]{geometry}
\setlength{\emergencystretch}{3em} % prevent overfull lines
\setcounter{secnumdepth}{-\maxdimen} % remove section numbering
% Make \paragraph and \subparagraph free-standing
\makeatletter
\ifx\paragraph\undefined\else
  \let\oldparagraph\paragraph
  \renewcommand{\paragraph}{
    \@ifstar
      \xxxParagraphStar
      \xxxParagraphNoStar
  }
  \newcommand{\xxxParagraphStar}[1]{\oldparagraph*{#1}\mbox{}}
  \newcommand{\xxxParagraphNoStar}[1]{\oldparagraph{#1}\mbox{}}
\fi
\ifx\subparagraph\undefined\else
  \let\oldsubparagraph\subparagraph
  \renewcommand{\subparagraph}{
    \@ifstar
      \xxxSubParagraphStar
      \xxxSubParagraphNoStar
  }
  \newcommand{\xxxSubParagraphStar}[1]{\oldsubparagraph*{#1}\mbox{}}
  \newcommand{\xxxSubParagraphNoStar}[1]{\oldsubparagraph{#1}\mbox{}}
\fi
\makeatother


\providecommand{\tightlist}{%
  \setlength{\itemsep}{0pt}\setlength{\parskip}{0pt}}\usepackage{longtable,booktabs,array}
\usepackage{calc} % for calculating minipage widths
% Correct order of tables after \paragraph or \subparagraph
\usepackage{etoolbox}
\makeatletter
\patchcmd\longtable{\par}{\if@noskipsec\mbox{}\fi\par}{}{}
\makeatother
% Allow footnotes in longtable head/foot
\IfFileExists{footnotehyper.sty}{\usepackage{footnotehyper}}{\usepackage{footnote}}
\makesavenoteenv{longtable}
\usepackage{graphicx}
\makeatletter
\newsavebox\pandoc@box
\newcommand*\pandocbounded[1]{% scales image to fit in text height/width
  \sbox\pandoc@box{#1}%
  \Gscale@div\@tempa{\textheight}{\dimexpr\ht\pandoc@box+\dp\pandoc@box\relax}%
  \Gscale@div\@tempb{\linewidth}{\wd\pandoc@box}%
  \ifdim\@tempb\p@<\@tempa\p@\let\@tempa\@tempb\fi% select the smaller of both
  \ifdim\@tempa\p@<\p@\scalebox{\@tempa}{\usebox\pandoc@box}%
  \else\usebox{\pandoc@box}%
  \fi%
}
% Set default figure placement to htbp
\def\fps@figure{htbp}
\makeatother
% definitions for citeproc citations
\NewDocumentCommand\citeproctext{}{}
\NewDocumentCommand\citeproc{mm}{%
  \begingroup\def\citeproctext{#2}\cite{#1}\endgroup}
\makeatletter
 % allow citations to break across lines
 \let\@cite@ofmt\@firstofone
 % avoid brackets around text for \cite:
 \def\@biblabel#1{}
 \def\@cite#1#2{{#1\if@tempswa , #2\fi}}
\makeatother
\newlength{\cslhangindent}
\setlength{\cslhangindent}{1.5em}
\newlength{\csllabelwidth}
\setlength{\csllabelwidth}{3em}
\newenvironment{CSLReferences}[2] % #1 hanging-indent, #2 entry-spacing
 {\begin{list}{}{%
  \setlength{\itemindent}{0pt}
  \setlength{\leftmargin}{0pt}
  \setlength{\parsep}{0pt}
  % turn on hanging indent if param 1 is 1
  \ifodd #1
   \setlength{\leftmargin}{\cslhangindent}
   \setlength{\itemindent}{-1\cslhangindent}
  \fi
  % set entry spacing
  \setlength{\itemsep}{#2\baselineskip}}}
 {\end{list}}
\usepackage{calc}
\newcommand{\CSLBlock}[1]{\hfill\break\parbox[t]{\linewidth}{\strut\ignorespaces#1\strut}}
\newcommand{\CSLLeftMargin}[1]{\parbox[t]{\csllabelwidth}{\strut#1\strut}}
\newcommand{\CSLRightInline}[1]{\parbox[t]{\linewidth - \csllabelwidth}{\strut#1\strut}}
\newcommand{\CSLIndent}[1]{\hspace{\cslhangindent}#1}

\usepackage{amsmath}
\usepackage{amssymb}
\usepackage{hyperref}
\makeatletter
\@ifpackageloaded{caption}{}{\usepackage{caption}}
\AtBeginDocument{%
\ifdefined\contentsname
  \renewcommand*\contentsname{Table of contents}
\else
  \newcommand\contentsname{Table of contents}
\fi
\ifdefined\listfigurename
  \renewcommand*\listfigurename{List of Figures}
\else
  \newcommand\listfigurename{List of Figures}
\fi
\ifdefined\listtablename
  \renewcommand*\listtablename{List of Tables}
\else
  \newcommand\listtablename{List of Tables}
\fi
\ifdefined\figurename
  \renewcommand*\figurename{Figure}
\else
  \newcommand\figurename{Figure}
\fi
\ifdefined\tablename
  \renewcommand*\tablename{Table}
\else
  \newcommand\tablename{Table}
\fi
}
\@ifpackageloaded{float}{}{\usepackage{float}}
\floatstyle{ruled}
\@ifundefined{c@chapter}{\newfloat{codelisting}{h}{lop}}{\newfloat{codelisting}{h}{lop}[chapter]}
\floatname{codelisting}{Listing}
\newcommand*\listoflistings{\listof{codelisting}{List of Listings}}
\makeatother
\makeatletter
\makeatother
\makeatletter
\@ifpackageloaded{caption}{}{\usepackage{caption}}
\@ifpackageloaded{subcaption}{}{\usepackage{subcaption}}
\makeatother

\usepackage{bookmark}

\IfFileExists{xurl.sty}{\usepackage{xurl}}{} % add URL line breaks if available
\urlstyle{same} % disable monospaced font for URLs
\hypersetup{
  pdftitle={Epidynamix: From Force to Field in Real-World Epidemiology},
  pdfauthor={Jinseob Kim},
  pdfkeywords={causal inference, real-world data, positivity
violation, treatment effect heterogeneity, risk potential field},
  colorlinks=true,
  linkcolor={blue},
  filecolor={Maroon},
  citecolor={Blue},
  urlcolor={Blue},
  pdfcreator={LaTeX via pandoc}}


\title{Epidynamix: From Force to Field in Real-World Epidemiology}
\author{Jinseob Kim}
\date{}

\begin{document}
\maketitle
\begin{abstract}
Standard causal inference methods estimate average treatment
effects---scalars that work well when effects are homogeneous and
positivity holds. In Real-World Data (RWD), however, clinical guidelines
create structural positivity violations, and effects vary systematically
across patient states. We propose Epidynamix, a complementary output
format that maps treatment effect heterogeneity explicitly. Defining a
Risk Potential Field \(\Phi_a(x) = -\log\lambda(x,a)\) for each
treatment regime, we compute the effect surface
\(\Delta_A\Phi(x) = \Phi_1(x) - \Phi_0(x)\)---mathematically equivalent
to CATE on the log-hazard scale. The key difference is interpretive:
positivity violations become structural boundaries to report, not
estimation failures to fix. Through simulation, we show that while Cox
regression and IPTW yield valid averages (HR = 0.55), the effect surface
reveals \textasciitilde3.5-fold variation across state space, with the
largest benefits in high-risk patients where guidelines mandate
treatment. This reframing does not replace causal inference but enriches
its output: \textbf{a map, not just a number.}
\end{abstract}


\setstretch{1.5}
\section{Introduction: When Averages Are Not
Enough}\label{introduction-when-averages-are-not-enough}

Modern causal inference in epidemiology typically asks:

\[
\text{ATE} = \mathbb{E}[Y(1) - Y(0)]
\]

This formulation assumes: well-defined potential outcomes \(Y(a)\),
interventions acting as external causes, and counterfactual symmetry
between \(a=0\) and \(a=1\). This works extremely well in randomized
controlled trials.

However, in Real-World Data (RWD), the system is high-dimensional,
time-varying, policy- and guideline-constrained, and
observer-influenced. Yet we still model intervention as a force:

\[
A \;\longrightarrow\; Y
\]

\textbf{Why now?} The explosion of RWD---electronic health records,
claims databases, registries---has exposed the limitations of this
paradigm. Unlike curated trial populations, RWD reflects the full
complexity of clinical practice: treatment decisions are constrained by
guidelines, contraindications create structural zeros in the propensity
score, and effects vary systematically across patient states (Petersen
et al. 2012). We increasingly face situations where:

\begin{itemize}
\tightlist
\item
  Treatment is mandated for high-risk patients (structural positivity
  violation)
\item
  The ``same'' intervention varies by context, timing, and patient state
\item
  Treatment decisions are entangled with prognosis (confounding by
  indication)
\end{itemize}

Traditional methods respond by restricting analysis to ``overlap
regions'' or trimming extreme weights---effectively discarding the very
patients for whom treatment decisions matter most.

We propose an alternative: \textbf{a map, not a number.}

\section{The Geometric Turn: From Force to
Field}\label{the-geometric-turn-from-force-to-field}

Physics offers a useful analogy. In Newtonian mechanics, gravity is a
force acting between masses. In general relativity, gravity emerges from
the geometry of spacetime---objects follow geodesics in a curved
manifold. This is not merely a notational change; it enabled new
predictions (light bending, gravitational waves).

We propose a \emph{conceptual reframing}, not a paradigm shift of equal
magnitude. The analogy is motivational, not literal:

\begin{itemize}
\tightlist
\item
  View treatment effects as properties of a structured state space
\item
  Represent heterogeneity as geometry rather than residual variance
\item
  Treat structural constraints as boundaries to be mapped, not
  estimation failures
\end{itemize}

This reframing does not generate new predictions from the same data. Its
value lies in changing \emph{what we report}: a map of effect
heterogeneity and structural boundaries, rather than a single number.

\subsection{Notation}\label{notation}

Throughout this paper, we use the following conventions (discrete
treatment):

\begin{longtable}[]{@{}
  >{\raggedright\arraybackslash}p{(\linewidth - 2\tabcolsep) * \real{0.4706}}
  >{\raggedright\arraybackslash}p{(\linewidth - 2\tabcolsep) * \real{0.5294}}@{}}
\toprule\noalign{}
\begin{minipage}[b]{\linewidth}\raggedright
Symbol
\end{minipage} & \begin{minipage}[b]{\linewidth}\raggedright
Meaning
\end{minipage} \\
\midrule\noalign{}
\endhead
\bottomrule\noalign{}
\endlastfoot
\(S_t = X_t\) & System state (covariates) at time \(t\) \\
\(\mathcal{S}\) & State space for \(X\) \\
\(\lambda(x, a, t)\) & Instantaneous hazard at state \(x\) under \(a\)
at time \(t\) \\
\(\Phi_a(x)\) & Risk potential under regime \(a\in\{0,1\}\):
\(-\log \lambda(x,a, t^*)\) or cumulative variant \\
\(\nabla_x \Phi_a\) & Gradient w.r.t. continuous covariates \(x\) (no
derivative in \(a\)) \\
\(\Delta_A \Phi(x)\) & Finite difference across regimes:
\(\Phi_1(x) - \Phi_0(x)\) \\
\(K_a(x_t, x_{t+1})\) & Transition kernel conditional on \(a_t\) \\
\(\mathcal{I}_a\) & Regime selection: \(x \mapsto \Phi_a(x)\) \\
\end{longtable}

Notes: - We treat \(a\) as a regime index, not as a coordinate of the
state. For binary \(a\), treatment-axis changes use finite differences
\(\Delta_A \Phi\) rather than derivatives. - Time handling is made
explicit via \(t\) or a chosen reference \(t^*\) (see Time Treatment).

\section{State Space and Risk Field}\label{state-space-and-risk-field}

\subsection{State Definition}\label{state-definition}

We define the observable system state as covariates only:

\[
S_t = X_t \in \mathcal{S}.
\]

Treatment \(A_t\) is a regime index applied to the system, not a
coordinate of \(X_t\). The dynamics of covariates can depend on
treatment via a treatment-specific transition kernel:

\[
P(X_{t+1} \mid X_t, A_t=a) = K_a(X_t, X_{t+1}).
\]

Regions where \(K_a\) assigns zero probability are \textbf{structural}
(e.g., guideline-constrained), not estimation failures.

\subsection{Risk Potential Function}\label{risk-potential-function}

Define a local hazard field \(\lambda(x,a,t)\) and the corresponding
potential:

\[
\Phi_a(x) = -\log\lambda(x,a,t^*)\quad\text{(landmark)}\qquad \text{or}\qquad \Phi_a(x) = -\log\int_0^\tau \lambda(x,a,t)\,dt\;\text{(cumulative)}.
\]

Interpretation: low \(\Phi_a\) → high risk; high \(\Phi_a\) → relative
stability. \(\Phi_a\) is a \textbf{geometric property of the state space
under regime \(a\)}.

\subsection{Time Treatment}\label{time-treatment}

RWD hazards are typically time-varying. We use one of:

\begin{enumerate}
\def\labelenumi{\arabic{enumi}.}
\tightlist
\item
  Landmark: \(\Phi_a(x) := -\log \lambda(x,a, t^*)\) at a clinically
  meaningful \(t^*\)
\item
  Discrete-time: estimate \(\lambda_t(x,a)\) on bins
  \(t\in\{1,\dots,T\}\) and summarize \(\Phi_{a,t}(x)\)
\item
  Cumulative: \(\Phi_a(x) := -\log \int_0^\tau \lambda(x,a,t)\,dt\)
\end{enumerate}

Examples use (1) for simplicity. For applied analyses, we recommend (1)
or (2), reporting \(t\) explicitly and avoiding a global stationarity
assumption.

\section{Intervention as Regime
Selection}\label{intervention-as-regime-selection}

Instead of viewing \(\text{do}(A=a)\) as an external force, we treat
intervention as selecting which potential surface to evaluate:

\[
\mathcal{I}_a : x \mapsto \Phi_a(x).
\]

This is not a state transformation but a regime selection---choosing
which of the two surfaces \(\{\Phi_0, \Phi_1\}\) applies to a given
covariate configuration \(x\).

\section{Vector Field Interpretation}\label{vector-field-interpretation}

\subsection{Gradient of the Potential}\label{gradient-of-the-potential}

Each \(\Phi_a(x)\) defines a scalar field on \(\mathcal{S}\). Its
spatial gradient defines a \textbf{vector field} over \(x\):

\[
\nabla_x \Phi_a(x) = \left( \frac{\partial \Phi_a}{\partial x_1}, \ldots, \frac{\partial \Phi_a}{\partial x_p} \right).
\]

For binary treatment, changes across \(a\) use the finite difference
\(\Delta_A\Phi(x) = \Phi_1(x) - \Phi_0(x)\); no derivative in \(a\) is
implied.

\textbf{Connection to Cox regression:} In a log-linear Cox model,
\(\log \lambda(x,a,t)=\log h_0(t)+\beta_x' x + \beta_A a\), so
\(\nabla_x \Phi_a(x) = -\beta_x\). The field framework generalizes this
to nonlinear \(\Phi_a(x)\).

\subsection{Treatment Contrast as Surface
Difference}\label{treatment-contrast-as-surface-difference}

For a given covariate configuration \(x\), the treatment effect is the
difference between the two potential surfaces:

\[
\Delta_A \Phi(x) = \Phi_1(x) - \Phi_0(x).
\]

Interpretation:

\begin{itemize}
\tightlist
\item
  \textbf{Protective effect:} \(\Delta_A \Phi(x) > 0\) (treatment lowers
  hazard)
\item
  \textbf{Harmful effect:} \(\Delta_A \Phi(x) < 0\) (treatment increases
  hazard)
\end{itemize}

This effect varies across \(x\)---it is a surface, not a scalar. The
spatial gradient \(\nabla_x \Phi_a\) describes how risk changes with
covariates within a fixed regime.

\section{ATE as a First-Order
Projection}\label{ate-as-a-first-order-projection}

The true objects are the potential surfaces \(\Phi_0(x)\) and
\(\Phi_1(x)\).

\textbf{For binary treatment} \(A \in \{0,1\}\), the treatment-contrast
surface is:

\[
\Delta_A \Phi(x) = \Phi_1(x) - \Phi_0(x).
\]

Averaging over \(X\) yields the ATE:

\[
\text{ATE} = \mathbb{E}_X[\Delta_A \Phi(X)].
\]

On the survival scale,
\(\Delta_A \Phi(x) = \log\{\lambda(x,0)/\lambda(x,1)\}\) is the log
hazard-ratio field; the HR field is \(\exp\{-\Delta_A \Phi(x)\}\).

Identification requires (i) exchangeability \(Y^a \perp A \mid X\) and
(ii) positivity on the target domain. Where positivity fails,
\(\Phi_0(x)\) or \(\Phi_1(x)\) is undefined, and \(\Delta_A\Phi(x)\) is
not identified---these regions are reported as structural boundaries.

\textbf{Case 1: Homogeneous effects}
(\(\text{Var}_X[\Delta_A \Phi] \approx 0\))

When \(\Delta_A \Phi(x) \approx c\) (constant across \(x\)), ATE
\(\approx c\). No information is lost.

\textbf{Case 2: Heterogeneous effects} (\(\text{Var}_X[\Delta_A \Phi]\)
large)

ATE remains well-defined but collapses a distribution into a single
number, losing information about where effects are strong vs weak,
positive vs negative.

\textbf{Case 3: Positivity violation} (structural discontinuity)

When \(P(A = 0 \mid X = x) = 0\) for some \(x\), the surface
\(\Phi_0(x)\) is undefined in that region. ATE involves integration over
regions where \(\Phi_0(x)\) does not exist. This is not an estimation
problem---it is a \textbf{domain problem}.

\section{Identification and
Positivity}\label{identification-and-positivity}

We do not avoid counterfactual reasoning; we avoid unnecessary notation.
Estimation of \(\Phi_a(x)\) and \(\Delta_A\Phi(x)\) still relies on
exchangeability and positivity on the support of \(X\). Where
\(P(A=a\mid X=x)=0\), \(\Phi_a(x)\) is not defined---this is a domain
issue. In the field view, such regions are \textbf{disconnected} parts
of \(\mathcal{S}\) and are mapped as structural boundaries rather than
imputed.

\section{Relation to MSM and the
g-formula}\label{relation-to-msm-and-the-g-formula}

\subsection{The g-formula as Integrated
Flow}\label{the-g-formula-as-integrated-flow}

The g-formula computes the expected accumulation of local risk along
trajectories constrained to follow a specific path through state space.

\subsection{MSMs as Average
Projections}\label{msms-as-average-projections}

MSMs estimate the average directional flow of the system through the
risk field, projected onto the treatment axis (Cole and Hernán 2008).

\subsection{Weight Instability as a Geometric
Signal}\label{weight-instability-as-a-geometric-signal}

When positivity is violated, inverse probability weights diverge.
Geometrically, this corresponds to disconnected regions where admissible
trajectories do not exist. Weight explosion is a signal that the causal
projection is being forced across regions with no valid geometric
connection.

\section{Simulation Study}\label{simulation-study}

We simulated a 2D state space with survival outcomes:

\begin{itemize}
\tightlist
\item
  \(X_1\): Systolic blood pressure (100--200 mmHg)
\item
  \(X_2\): Inflammatory marker (0--10 mg/L)
\item
  \(A\): Binary treatment (antihypertensive)
\item
  Structural constraint: If \(X_1 > 160\), then \(A = 1\) (mandatory per
  guideline)
\item
  Heterogeneous effect: Treatment benefit increases with \(X_1\)
\end{itemize}

\textbf{Estimation notes:} We enforce the structural rule
\(X_1 > 160 \Rightarrow A = 1\) deterministically to induce a structural
positivity boundary. We estimate the hazard surfaces
\(\hat{\lambda}(x,a)\) using generalized additive models (GAMs) with
tensor product smooths, fitting separate surfaces for \(A=0\) and
\(A=1\). From these, we obtain
\(\hat{\Phi}_a(x) = -\log \hat{\lambda}(x,a)\) and compute
\(\Delta_A \Phi(x) = \hat{\Phi}_1(x) - \hat{\Phi}_0(x)\). We report both
\(\Delta_A \Phi\) and the HR field \(\exp\{-\Delta_A \Phi\}\).
Visualization is restricted to regions with empirical support;
disconnected regions (no joint support for both \(A=0\) and \(A=1\)) are
rendered as boundaries where \(\Delta_A \Phi\) is undefined. Alternative
estimators (e.g., neural networks, random survival forests) could be
substituted; the conceptual framework does not depend on a specific
method.

\textbf{Results (N = 3,000):}

\begin{longtable}[]{@{}
  >{\raggedright\arraybackslash}p{(\linewidth - 4\tabcolsep) * \real{0.2500}}
  >{\raggedright\arraybackslash}p{(\linewidth - 4\tabcolsep) * \real{0.2500}}
  >{\raggedright\arraybackslash}p{(\linewidth - 4\tabcolsep) * \real{0.5000}}@{}}
\toprule\noalign{}
\begin{minipage}[b]{\linewidth}\raggedright
Method
\end{minipage} & \begin{minipage}[b]{\linewidth}\raggedright
Output
\end{minipage} & \begin{minipage}[b]{\linewidth}\raggedright
Interpretation
\end{minipage} \\
\midrule\noalign{}
\endhead
\bottomrule\noalign{}
\endlastfoot
Cox HR & 0.55 & Treatment reduces hazard by 45\% \\
IPTW ATE & 0.20 & Treatment increases 1-year survival by 20 percentage
points \\
HR field \(\exp(-\Delta_A \Phi)\) & 0.19--0.66 & Effect varies
\textasciitilde3.5-fold across state space \\
\end{longtable}

Traditional methods produce single numbers. The Field approach reveals
that treatment benefit varies \textasciitilde3.5-fold across the state
space, with the largest effects in high-BP patients---precisely those
for whom the guideline mandates treatment.

\begin{figure}

\centering{

\includegraphics[width=0.9\linewidth,height=\textheight,keepaspectratio]{fig_field_approach.png}

}

\caption{\label{fig-field}Treatment effect field showing heterogeneity
and structural boundary at \(X_1 > 160\). Gray region indicates
positivity violation where comparison is structurally impossible.}

\end{figure}%

\textbf{Key Messages:}

\begin{enumerate}
\def\labelenumi{\arabic{enumi}.}
\tightlist
\item
  Traditional methods are not wrong---Cox and IPTW correctly estimate
  average effects
\item
  But averages hide structure---HR = 0.55 does not reveal
  \textasciitilde3.5-fold variation
\item
  Positivity violation is structural---the gray zone reflects clinical
  logic where \(\Phi_a(x)\) is undefined
\item
  Field-based output is a map, not a number
\end{enumerate}

\section{Estimation Details
(Practical)}\label{estimation-details-practical}

\begin{itemize}
\tightlist
\item
  Model class: Flexible regressors for \(\log\lambda(x,a,t)\) (e.g.,
  GAMs with additive smooths and selected interactions: \texttt{s(x\_j)}
  and \texttt{ti(x\_j,x\_k)}) fitted separately by \(a\) or jointly with
  interaction terms.
\item
  Smoothing: Choose basis size \(k\) via REML or cross-validation;
  prefer lower-order interactions to mitigate the curse of
  dimensionality. For \(p\gg 2\), use (i) additive structures with
  sparse interactions, (ii) dimension reduction (PCA/PLS on \(X\)), or
  (iii) prior feature screening to stabilize surface estimation.
\item
  Variance: Obtain uncertainty via nonparametric bootstrap over
  individuals and refit surfaces; propagate to \(\Delta_A\Phi(x)\)
  pointwise. For GAMs, extract covariance of smooths for delta-method
  approximations where feasible.
\item
  Sample size: For dense 2D surfaces (\(k\approx 15\) per axis),
  \(N\geq \mathcal{O}(10^3)\) is typically needed for stable maps;
  higher dimensions require additive structures and stronger
  regularization.
\item
  Reporting: Restrict visualization to regions with empirical support
  for both \(a=0,1\); mask others and report as structural boundaries.
\item
  Boundary detection: (i) define common support as
  \(\{x: n(a=0|x)\geq m \text{ and } n(a=1|x)\geq m\}\) or via
  propensity score trimming
  \(\hat{e}(x)\in[\varepsilon, 1-\varepsilon]\); (ii) mask non-supported
  regions (NA) and render as gray in visualizations; (iii) report
  sensitivity to the threshold \(\varepsilon\) or minimum count \(m\).
\end{itemize}

\section{Dynamics and the Transition
Kernel}\label{dynamics-and-the-transition-kernel}

For time-varying treatment, write
\(P(X_{t+1}\mid X_t, A_t=a)=K_a(X_t,X_{t+1})\). Policy constraints
appear as zeros of \(K_a\). Regime contrasts over horizons can be
summarized by integrating hazards along admissible trajectories
(discrete-time g-formula) and visualized as time-indexed fields
\(\Phi_{a,t}(x)\).

\section{Relation to Existing Heterogeneity
Research}\label{relation-to-existing-heterogeneity-research}

\subsection{Differentiation from CATE}\label{differentiation-from-cate}

\begin{longtable}[]{@{}
  >{\raggedright\arraybackslash}p{(\linewidth - 4\tabcolsep) * \real{0.1951}}
  >{\raggedright\arraybackslash}p{(\linewidth - 4\tabcolsep) * \real{0.5122}}
  >{\raggedright\arraybackslash}p{(\linewidth - 4\tabcolsep) * \real{0.2927}}@{}}
\toprule\noalign{}
\begin{minipage}[b]{\linewidth}\raggedright
Aspect
\end{minipage} & \begin{minipage}[b]{\linewidth}\raggedright
CATE / HTE Research
\end{minipage} & \begin{minipage}[b]{\linewidth}\raggedright
Epidynamix
\end{minipage} \\
\midrule\noalign{}
\endhead
\bottomrule\noalign{}
\endlastfoot
Core question & ``What is the effect for subgroup \(X=x\)?'' & ``What is
the structure of the risk field?'' \\
Output & \(\tau(x) = \mathbb{E}[Y(1) - Y(0) \mid X=x]\) &
\(\Delta_A\Phi(x)\), structural boundaries \\
Positivity violation & Estimation problem (trim, extrapolate) &
\textbf{Information} (map the boundary) \\
Counterfactuals & Explicit potential outcomes & Same logic, different
framing \\
Goal & Better effect estimation & Richer output format \\
\end{longtable}

\textbf{Mathematical equivalence:} For survival outcomes,
\(\Delta_A\Phi(x) = \log\{\lambda(x,0)/\lambda(x,1)\}\) is
mathematically equivalent to CATE on the log-hazard scale. We do not
claim a new estimand.

\textbf{Key distinction:} The difference is in output format and
interpretation of failures. CATE treats positivity violations as missing
data to be handled (trimming, extrapolation); Epidynamix reports these
boundaries explicitly as structural features of the clinical landscape.
Both require exchangeability; the field framing simply elevates
heterogeneity and boundaries from nuisance to primary output.

\subsection{Connection to Structural Nested
Models}\label{connection-to-structural-nested-models}

The local effect \(\Delta_A \Phi(x)\) shares conceptual ground with the
``blip function'' in Structural Nested Mean Models (Robins 1994), which
also models treatment effects as functions of patient state. The key
difference is interpretive: SNMM aims to estimate causal effects under
sequential exchangeability, while Epidynamix treats the effect surface
as a geometric object to be mapped.

\subsection{Novelty Claim}\label{novelty-claim}

We do not claim new mathematics or a new estimand. \(\Delta_A\Phi(x)\)
is CATE on a transformed scale. Potential landscapes exist in physics
and systems biology (e.g., Waddington's epigenetic landscape (Waddington
1957)). Concurrent work by Leizerman (Leizerman 2025) develops a
``Unified Causal Field Theory'' using differential geometry---a more
general formalization.

Our contribution is practical and interpretive:

\begin{enumerate}
\def\labelenumi{\arabic{enumi}.}
\tightlist
\item
  \textbf{Output format:} Report the effect surface, not just its
  average
\item
  \textbf{Boundary elevation:} Treat positivity violations as structural
  findings, not nuisance
\item
  \textbf{Clinical RWD focus:} Where guidelines create structural zeros,
  mapping boundaries may be more informative than forcing estimation
\end{enumerate}

This is a \textbf{reframing of output}, not a methodological invention.

\section{Discussion}\label{discussion}

The Epidynamix framework offers a complementary perspective to standard
causal inference (Rubin 1974; Hernán and Robins 2020). It does not
replace existing methods; rather, it clarifies their domain of validity.

\textbf{When ATE works:} If the risk field is smooth, low-curvature, and
well-connected (positivity holds everywhere), the ATE is an accurate
summary.

\textbf{When ATE fails:} If effects are highly heterogeneous, or if
structural constraints create disconnected regions, the ATE may be
misleading or undefined. The field approach provides richer output---a
map rather than a number.

\textbf{Practical implications:}

\begin{enumerate}
\def\labelenumi{\arabic{enumi}.}
\tightlist
\item
  Report effect heterogeneity alongside average effects
\item
  Visualize the treatment effect landscape when possible
\item
  Treat positivity violations as findings, not errors
\item
  Consider whether the causal question is answerable before estimating
\item
  Report HR field maps (\(\exp\{-\Delta_A \Phi\}\)) alongside scalar
  summaries
\end{enumerate}

\textbf{Limitations:} The framework is conceptual; practical estimation
of \(\Phi_a(x)\) requires flexible models (e.g., GAMs, neural networks)
and careful validation. Extension to time-varying treatments and
high-dimensional states remains an open challenge.

\section{Conclusion}\label{conclusion}

Standard causal inference methods are not wrong---they provide valid
average effect estimates under their assumptions. However, when effects
are highly heterogeneous or structural constraints create positivity
violations, a scalar summary may be insufficient.

The Epidynamix framework offers a complementary output format: mapping
the treatment effect surface \(\Delta_A\Phi(x)\) and explicitly
reporting regions where causal comparisons are structurally impossible.
This is not a new estimand but a richer way to present heterogeneity and
structural boundaries.

When averages hide important structure, consider reporting the map
alongside the number.

\section{References}\label{references}

\phantomsection\label{refs}
\begin{CSLReferences}{1}{0}
\bibitem[\citeproctext]{ref-cole2008}
Cole, Stephen R, and Miguel A Hernán. 2008. {``Constructing Inverse
Probability Weights for Marginal Structural Models.''} \emph{American
Journal of Epidemiology} 168 (6): 656--64.

\bibitem[\citeproctext]{ref-hernan2020}
Hernán, Miguel A, and James M Robins. 2020. \emph{Causal Inference: What
If}. Boca Raton: Chapman \& Hall/CRC.

\bibitem[\citeproctext]{ref-leizerman2025}
Leizerman, Samuel L. 2025. {``Unified Causal Field Theory: A Proof of
Geometric Subsumption and Extension of Causal Inference Methods into
Unified Framework.''} SocArXiv.
\url{https://doi.org/10.31235/osf.io/c7pz9}.

\bibitem[\citeproctext]{ref-petersen2012}
Petersen, Maya L, Kristin E Porter, Susan Gruber, Yue Wang, and Mark J
van der Laan. 2012. {``Diagnosing and Responding to Violations in the
Positivity Assumption.''} \emph{Statistical Methods in Medical Research}
21 (1): 31--54. \url{https://doi.org/10.1177/0962280210386207}.

\bibitem[\citeproctext]{ref-robins1994}
Robins, James M. 1994. {``Correcting for Non-Compliance in Randomized
Trials Using Structural Nested Mean Models.''} \emph{Communications in
Statistics-Theory and Methods} 23 (8): 2379--2412.

\bibitem[\citeproctext]{ref-rubin1974}
Rubin, Donald B. 1974. {``Estimating Causal Effects of Treatments in
Randomized and Nonrandomized Studies.''} \emph{Journal of Educational
Psychology} 66 (5): 688.

\bibitem[\citeproctext]{ref-waddington1957}
Waddington, Conrad Hal. 1957. \emph{The Strategy of the Genes: A
Discussion of Some Aspects of Theoretical Biology}. London: George Allen
\& Unwin.

\end{CSLReferences}




\end{document}
