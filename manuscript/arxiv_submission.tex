% Options for packages loaded elsewhere
\PassOptionsToPackage{unicode}{hyperref}
\PassOptionsToPackage{hyphens}{url}
\PassOptionsToPackage{dvipsnames,svgnames,x11names}{xcolor}
%
\documentclass[
  11pt,
]{article}

\usepackage{amsmath,amssymb}
\usepackage{setspace}
\usepackage{iftex}
\ifPDFTeX
  \usepackage[T1]{fontenc}
  \usepackage[utf8]{inputenc}
  \usepackage{textcomp} % provide euro and other symbols
\else % if luatex or xetex
  \usepackage{unicode-math}
  \defaultfontfeatures{Scale=MatchLowercase}
  \defaultfontfeatures[\rmfamily]{Ligatures=TeX,Scale=1}
\fi
\usepackage{lmodern}
\ifPDFTeX\else  
    % xetex/luatex font selection
\fi
% Use upquote if available, for straight quotes in verbatim environments
\IfFileExists{upquote.sty}{\usepackage{upquote}}{}
\IfFileExists{microtype.sty}{% use microtype if available
  \usepackage[]{microtype}
  \UseMicrotypeSet[protrusion]{basicmath} % disable protrusion for tt fonts
}{}
\makeatletter
\@ifundefined{KOMAClassName}{% if non-KOMA class
  \IfFileExists{parskip.sty}{%
    \usepackage{parskip}
  }{% else
    \setlength{\parindent}{0pt}
    \setlength{\parskip}{6pt plus 2pt minus 1pt}}
}{% if KOMA class
  \KOMAoptions{parskip=half}}
\makeatother
\usepackage{xcolor}
\usepackage[margin=1in]{geometry}
\setlength{\emergencystretch}{3em} % prevent overfull lines
\setcounter{secnumdepth}{-\maxdimen} % remove section numbering
% Make \paragraph and \subparagraph free-standing
\makeatletter
\ifx\paragraph\undefined\else
  \let\oldparagraph\paragraph
  \renewcommand{\paragraph}{
    \@ifstar
      \xxxParagraphStar
      \xxxParagraphNoStar
  }
  \newcommand{\xxxParagraphStar}[1]{\oldparagraph*{#1}\mbox{}}
  \newcommand{\xxxParagraphNoStar}[1]{\oldparagraph{#1}\mbox{}}
\fi
\ifx\subparagraph\undefined\else
  \let\oldsubparagraph\subparagraph
  \renewcommand{\subparagraph}{
    \@ifstar
      \xxxSubParagraphStar
      \xxxSubParagraphNoStar
  }
  \newcommand{\xxxSubParagraphStar}[1]{\oldsubparagraph*{#1}\mbox{}}
  \newcommand{\xxxSubParagraphNoStar}[1]{\oldsubparagraph{#1}\mbox{}}
\fi
\makeatother


\providecommand{\tightlist}{%
  \setlength{\itemsep}{0pt}\setlength{\parskip}{0pt}}\usepackage{longtable,booktabs,array}
\usepackage{calc} % for calculating minipage widths
% Correct order of tables after \paragraph or \subparagraph
\usepackage{etoolbox}
\makeatletter
\patchcmd\longtable{\par}{\if@noskipsec\mbox{}\fi\par}{}{}
\makeatother
% Allow footnotes in longtable head/foot
\IfFileExists{footnotehyper.sty}{\usepackage{footnotehyper}}{\usepackage{footnote}}
\makesavenoteenv{longtable}
\usepackage{graphicx}
\makeatletter
\newsavebox\pandoc@box
\newcommand*\pandocbounded[1]{% scales image to fit in text height/width
  \sbox\pandoc@box{#1}%
  \Gscale@div\@tempa{\textheight}{\dimexpr\ht\pandoc@box+\dp\pandoc@box\relax}%
  \Gscale@div\@tempb{\linewidth}{\wd\pandoc@box}%
  \ifdim\@tempb\p@<\@tempa\p@\let\@tempa\@tempb\fi% select the smaller of both
  \ifdim\@tempa\p@<\p@\scalebox{\@tempa}{\usebox\pandoc@box}%
  \else\usebox{\pandoc@box}%
  \fi%
}
% Set default figure placement to htbp
\def\fps@figure{htbp}
\makeatother
% definitions for citeproc citations
\NewDocumentCommand\citeproctext{}{}
\NewDocumentCommand\citeproc{mm}{%
  \begingroup\def\citeproctext{#2}\cite{#1}\endgroup}
\makeatletter
 % allow citations to break across lines
 \let\@cite@ofmt\@firstofone
 % avoid brackets around text for \cite:
 \def\@biblabel#1{}
 \def\@cite#1#2{{#1\if@tempswa , #2\fi}}
\makeatother
\newlength{\cslhangindent}
\setlength{\cslhangindent}{1.5em}
\newlength{\csllabelwidth}
\setlength{\csllabelwidth}{3em}
\newenvironment{CSLReferences}[2] % #1 hanging-indent, #2 entry-spacing
 {\begin{list}{}{%
  \setlength{\itemindent}{0pt}
  \setlength{\leftmargin}{0pt}
  \setlength{\parsep}{0pt}
  % turn on hanging indent if param 1 is 1
  \ifodd #1
   \setlength{\leftmargin}{\cslhangindent}
   \setlength{\itemindent}{-1\cslhangindent}
  \fi
  % set entry spacing
  \setlength{\itemsep}{#2\baselineskip}}}
 {\end{list}}
\usepackage{calc}
\newcommand{\CSLBlock}[1]{\hfill\break\parbox[t]{\linewidth}{\strut\ignorespaces#1\strut}}
\newcommand{\CSLLeftMargin}[1]{\parbox[t]{\csllabelwidth}{\strut#1\strut}}
\newcommand{\CSLRightInline}[1]{\parbox[t]{\linewidth - \csllabelwidth}{\strut#1\strut}}
\newcommand{\CSLIndent}[1]{\hspace{\cslhangindent}#1}

\usepackage{amsmath}
\usepackage{amssymb}
\usepackage{hyperref}
\makeatletter
\@ifpackageloaded{caption}{}{\usepackage{caption}}
\AtBeginDocument{%
\ifdefined\contentsname
  \renewcommand*\contentsname{Table of contents}
\else
  \newcommand\contentsname{Table of contents}
\fi
\ifdefined\listfigurename
  \renewcommand*\listfigurename{List of Figures}
\else
  \newcommand\listfigurename{List of Figures}
\fi
\ifdefined\listtablename
  \renewcommand*\listtablename{List of Tables}
\else
  \newcommand\listtablename{List of Tables}
\fi
\ifdefined\figurename
  \renewcommand*\figurename{Figure}
\else
  \newcommand\figurename{Figure}
\fi
\ifdefined\tablename
  \renewcommand*\tablename{Table}
\else
  \newcommand\tablename{Table}
\fi
}
\@ifpackageloaded{float}{}{\usepackage{float}}
\floatstyle{ruled}
\@ifundefined{c@chapter}{\newfloat{codelisting}{h}{lop}}{\newfloat{codelisting}{h}{lop}[chapter]}
\floatname{codelisting}{Listing}
\newcommand*\listoflistings{\listof{codelisting}{List of Listings}}
\makeatother
\makeatletter
\makeatother
\makeatletter
\@ifpackageloaded{caption}{}{\usepackage{caption}}
\@ifpackageloaded{subcaption}{}{\usepackage{subcaption}}
\makeatother

\usepackage{bookmark}

\IfFileExists{xurl.sty}{\usepackage{xurl}}{} % add URL line breaks if available
\urlstyle{same} % disable monospaced font for URLs
\hypersetup{
  pdftitle={Epidynamix: From Force to Field in Real-World Epidemiology},
  pdfauthor={Jinseob Kim},
  pdfkeywords={causal inference, real-world data, positivity
violation, treatment effect heterogeneity, risk potential field},
  colorlinks=true,
  linkcolor={blue},
  filecolor={Maroon},
  citecolor={Blue},
  urlcolor={Blue},
  pdfcreator={LaTeX via pandoc}}


\title{Epidynamix: From Force to Field in Real-World Epidemiology}
\author{Jinseob Kim}
\date{}

\begin{document}
\maketitle
\begin{abstract}
Modern causal inference in epidemiology treats interventions as external
forces acting on passive subjects to produce average outcomes---a
``Newtonian'' abstraction that works well in randomized trials but
falters in Real-World Data (RWD). In RWD, clinical guidelines create
structural positivity violations, treatment decisions are entangled with
prognosis, and effects are inherently heterogeneous across patient
states. We propose Epidynamix, a framework that shifts focus from
``Force'' to ``Field.'' By viewing health states as points within a Risk
Potential Field \(\Phi(s) = -\log\lambda(s)\), we reinterpret causal
effects as directional gradients, the ATE as a first-order projection,
and positivity violations as structural boundaries---not estimation
failures, but geometric features of the clinical landscape. Through
simulation, we demonstrate that while standard methods (Cox regression,
IPTW) collapse effect heterogeneity into scalar summaries, the
field-based approach preserves clinically meaningful variation
(\textasciitilde3.5-fold across state space) and explicitly maps regions
where comparison is structurally impossible. This geometric perspective
offers a complementary lens for RWD analysis: \textbf{a map, not a
number.}
\end{abstract}


\setstretch{1.5}
\section{Introduction: The Newtonian
Discomfort}\label{introduction-the-newtonian-discomfort}

Modern causal inference in epidemiology typically asks:

\[
\text{ATE} = \mathbb{E}[Y(1) - Y(0)]
\]

This formulation assumes: well-defined potential outcomes \(Y(a)\),
interventions acting as external causes, and counterfactual symmetry
between \(a=0\) and \(a=1\). This works extremely well in randomized
controlled trials.

However, in Real-World Data (RWD), the system is high-dimensional,
time-varying, policy- and guideline-constrained, and
observer-influenced. Yet we still model intervention as a force:

\[
A \;\longrightarrow\; Y
\]

\textbf{Why now?} The explosion of RWD---electronic health records,
claims databases, registries---has exposed the limitations of this
paradigm. Unlike curated trial populations, RWD reflects the full
complexity of clinical practice: treatment decisions are constrained by
guidelines, contraindications create structural zeros in the propensity
score, and effects vary systematically across patient states (Petersen
et al. 2012). We increasingly face situations where:

\begin{itemize}
\tightlist
\item
  Treatment is mandated for high-risk patients (structural positivity
  violation)
\item
  The ``same'' intervention varies by context, timing, and patient state
\item
  Treatment decisions are entangled with prognosis (confounding by
  indication)
\end{itemize}

Traditional methods respond by restricting analysis to ``overlap
regions'' or trimming extreme weights---effectively discarding the very
patients for whom treatment decisions matter most.

We propose an alternative: \textbf{a map, not a number.}

\section{The Geometric Turn: From Force to
Field}\label{the-geometric-turn-from-force-to-field}

Einstein did not reinterpret gravity by refining the force. He removed
the force. Objects do not respond to gravity---they follow geodesics in
a curved spacetime.

We propose an analogous shift:

\begin{itemize}
\tightlist
\item
  Causal effects are not primitive forces
\item
  Outcomes arise from motion within a structured field
\item
  Interventions reposition systems inside that field
\end{itemize}

\subsection{Notation}\label{notation}

Throughout this paper, we use the following conventions:

\begin{longtable}[]{@{}
  >{\raggedright\arraybackslash}p{(\linewidth - 2\tabcolsep) * \real{0.4706}}
  >{\raggedright\arraybackslash}p{(\linewidth - 2\tabcolsep) * \real{0.5294}}@{}}
\toprule\noalign{}
\begin{minipage}[b]{\linewidth}\raggedright
Symbol
\end{minipage} & \begin{minipage}[b]{\linewidth}\raggedright
Meaning
\end{minipage} \\
\midrule\noalign{}
\endhead
\bottomrule\noalign{}
\endlastfoot
\(S_t = (X_t, A_t)\) & System state at time \(t\) \\
\(\mathcal{S}\) & State space \\
\(\lambda(s)\) & Instantaneous hazard at state \(s\) \\
\(\Phi(s)\) & Risk potential function \\
\(\nabla \Phi\) & Gradient (for continuous variables) \\
\(\delta_A \Phi(x)\) & Finite difference for binary \(A\):
\(\Phi(x,1) - \Phi(x,0)\) \\
\(K(s, s')\) & Transition kernel \\
\(\mathcal{I}_a\) & Intervention operator \\
\end{longtable}

\textbf{Note on binary treatment:} When \(A \in \{0, 1\}\), expressions
like \(\partial \Phi / \partial a\) should be understood as the finite
difference \(\delta_A \Phi\), not a true derivative.

\section{State Space and Risk Field}\label{state-space-and-risk-field}

\subsection{State Definition}\label{state-definition}

We define the system state as:

\[
S_t = (X_t, A_t) \in \mathcal{S}
\]

where treatment \(A_t\) is a coordinate of the system, not an external
action. The system evolves according to a transition kernel:

\[
P(S_{t+1} \mid S_t) = K(S_t, S_{t+1})
\]

Regions where \(K\) assigns zero probability are \textbf{structural},
not violations.

\subsection{Risk Potential Function}\label{risk-potential-function}

Define a local risk field:

\[
\lambda(s) = \lim_{\Delta t \to 0} \frac{P(T \in [t, t+\Delta t) \mid S_t = s)}{\Delta t}
\]

This is estimable using standard survival models. We then define the
risk potential:

\[
\Phi(s) = -\log \lambda(s)
\]

Interpretation: low \(\Phi\) → high risk; high \(\Phi\) → relative
stability. \(\Phi\) is not an outcome---it is a \textbf{geometric
property of the state space}.

\subsection{Time Treatment}\label{time-treatment}

In this framework, we treat \(\Phi(s)\) as either:

\begin{enumerate}
\def\labelenumi{\arabic{enumi}.}
\tightlist
\item
  \textbf{Landmark approach:} Evaluated at a fixed reference time
  \(t^*\)
\item
  \textbf{Stationary assumption:} Time-invariant for a given state
\item
  \textbf{Cumulative formulation:} Integrated over a horizon
  \([0, \tau]\)
\end{enumerate}

For the examples in this paper, we adopt the \textbf{stationary
assumption} for simplicity.

In non-stationary hazard settings, consider landmark or time-specific
fields, or a cumulative definition of risk potential over \([0, \tau]\).

\section{Intervention as a Transition
Operator}\label{intervention-as-a-transition-operator}

Instead of a do-operator \(\text{do}(A=a)\), we define intervention as a
state transformation:

\[
\mathcal{I}_a : \mathcal{S} \to \mathcal{S}, \quad \mathcal{I}_a(x,a') = (x,a)
\]

Interventions move the system across the field. They do not directly
generate outcomes.

\section{Vector Field Interpretation}\label{vector-field-interpretation}

\subsection{Gradient of the Potential}\label{gradient-of-the-potential}

\(\Phi\) defines a scalar field on \(\mathcal{S}\). Its gradient defines
a \textbf{vector field}:

\[
\nabla \Phi(s) = \left( \frac{\partial \Phi}{\partial x_1}, \ldots, \frac{\partial \Phi}{\partial x_p}, \frac{\partial \Phi}{\partial a} \right)
\]

This \textbf{risk-gradient field} encodes directions of maximal risk
increase and local instability structure. For binary treatment,
interpret the treatment axis as finite differences \(\delta_A \Phi\)
rather than a true derivative.

\textbf{Connection to Cox regression:} In the Cox model,
\(\log \lambda(s) = \log h_0(t) + \beta' s\). Since
\(\Phi = -\log \lambda\), we have \(\nabla_x \Phi = -\beta\)
(independent of baseline hazard \(h_0\)). The field framework
generalizes this to nonlinear \(\Phi(s)\).

\subsection{Directional Effects as Local
Geometry}\label{directional-effects-as-local-geometry}

An intervention induces a local displacement
\(\Delta s = (0,\ldots,0,\Delta a)\). The induced change in potential
is:

\[
\Delta \Phi \approx \nabla \Phi(s) \cdot \Delta s
\]

Definitions:

\begin{itemize}
\tightlist
\item
  \textbf{Risk-increasing effect:} \(\Delta \Phi < 0\)
\item
  \textbf{Protective effect:} \(\Delta \Phi > 0\)
\end{itemize}

These effects are local, state-dependent, and directional. They are
\textbf{not global scalar quantities}.

\section{ATE as a First-Order
Projection}\label{ate-as-a-first-order-projection}

The true object is the field \(\Phi(x,a)\).

\textbf{For binary treatment} \(A \in \{0,1\}\), the treatment effect is
exactly the finite difference:

\[
\delta_A \Phi(x) = \Phi(x, 1) - \Phi(x, 0)
\]

Averaging over \(X\) yields the ATE:

\[
\text{ATE} = \mathbb{E}_X[\delta_A \Phi(X)]
\]

\textbf{ATE is a first-order projection of a high-dimensional geometry.}

\textbf{Survival scale note:} For survival outcomes,
\(\delta_A \Phi(x) = \Phi(x,1) - \Phi(x,0) = \log\{\lambda(x,0)/\lambda(x,1)\}\)
is the negative log hazard ratio. The HR field is
\(\exp\{-\delta_A \Phi(x)\}\).

\textbf{Case 1: Homogeneous effects}
(\(\text{Var}_X[\delta_A \Phi] \approx 0\))

When \(\delta_A \Phi(x) \approx c\) (constant across \(x\)), ATE
\(\approx c\). No information is lost.

\textbf{Case 2: Heterogeneous effects} (\(\text{Var}_X[\delta_A \Phi]\)
large)

ATE remains well-defined but collapses a distribution into a single
number, losing information about where effects are strong vs weak,
positive vs negative.

\textbf{Case 3: Positivity violation} (structural discontinuity)

When \(P(A = 0 \mid X = x) = 0\) for some \(x\), the field
\(\Phi(x, 0)\) is undefined in that region. ATE involves integration
over regions where \(\Phi(x, 0)\) does not exist. This is not an
estimation problem---it is a \textbf{domain problem}.

\section{Positivity Reinterpreted
Geometrically}\label{positivity-reinterpreted-geometrically}

Standard positivity requires \(0 < P(A=1 \mid X=x) < 1\).

In the field view: positivity violations correspond to
\textbf{disconnected regions} of \(\mathcal{S}\)---geometric
constraints, not estimation failures. Rather than treating these as
problems to fix, we map them as the defining boundaries of the clinical
landscape.

\section{Relation to MSM and the
g-formula}\label{relation-to-msm-and-the-g-formula}

\subsection{The g-formula as Integrated
Flow}\label{the-g-formula-as-integrated-flow}

The g-formula computes the expected accumulation of local risk along
trajectories constrained to follow a specific path through state space.

\subsection{MSMs as Average
Projections}\label{msms-as-average-projections}

MSMs estimate the average directional flow of the system through the
risk field, projected onto the treatment axis (Cole and Hernán 2008).

\subsection{Weight Instability as a Geometric
Signal}\label{weight-instability-as-a-geometric-signal}

When positivity is violated, inverse probability weights diverge.
Geometrically, this corresponds to disconnected regions where admissible
trajectories do not exist. Weight explosion is a signal that the causal
projection is being forced across regions with no valid geometric
connection.

\section{Simulation Study}\label{simulation-study}

We simulated a 2D state space with survival outcomes:

\begin{itemize}
\tightlist
\item
  \(X_1\): Systolic blood pressure (100--200 mmHg)
\item
  \(X_2\): Inflammatory marker (0--10 mg/L)
\item
  \(A\): Binary treatment (antihypertensive)
\item
  Structural constraint: If \(X_1 > 160\), then \(A = 1\) (mandatory per
  guideline)
\item
  Heterogeneous effect: Treatment benefit increases with \(X_1\)
\end{itemize}

\textbf{Estimation notes:} We enforce the structural rule
\(X_1 > 160 \Rightarrow A = 1\) deterministically to induce a structural
positivity boundary. We estimate the hazard surface \(\hat{\lambda}(s)\)
using generalized additive models (GAMs) with tensor product smooths,
fitting separate surfaces for \(A=0\) and \(A=1\). From these, we obtain
\(\hat{\Phi}(s) = -\log \hat{\lambda}(s)\) and compute
\(\delta_A \Phi(x) = \hat{\Phi}(x,1) - \hat{\Phi}(x,0)\). We report both
\(\delta_A \Phi\) and the HR field \(\exp\{-\delta_A \Phi\}\).
Visualization is restricted to regions with empirical support;
disconnected regions (no joint support for both \(A=0\) and \(A=1\)) are
rendered as boundaries where \(\delta_A \Phi\) is undefined. Alternative
estimators (e.g., neural networks, random survival forests) could be
substituted; the conceptual framework does not depend on a specific
method.

\textbf{Results (N = 3,000):}

\begin{longtable}[]{@{}
  >{\raggedright\arraybackslash}p{(\linewidth - 4\tabcolsep) * \real{0.2500}}
  >{\raggedright\arraybackslash}p{(\linewidth - 4\tabcolsep) * \real{0.2500}}
  >{\raggedright\arraybackslash}p{(\linewidth - 4\tabcolsep) * \real{0.5000}}@{}}
\toprule\noalign{}
\begin{minipage}[b]{\linewidth}\raggedright
Method
\end{minipage} & \begin{minipage}[b]{\linewidth}\raggedright
Output
\end{minipage} & \begin{minipage}[b]{\linewidth}\raggedright
Interpretation
\end{minipage} \\
\midrule\noalign{}
\endhead
\bottomrule\noalign{}
\endlastfoot
Cox HR & 0.55 & Treatment reduces hazard by 45\% \\
IPTW ATE & 0.20 & Treatment increases 1-year survival by 20 percentage
points \\
HR field \(\exp(-\delta_A \Phi)\) & 0.19--0.66 & Effect varies
\textasciitilde3.5-fold across state space \\
\end{longtable}

Traditional methods produce single numbers. The Field approach reveals
that treatment benefit varies \textasciitilde3.5-fold across the state
space, with the largest effects in high-BP patients---precisely those
for whom the guideline mandates treatment.

\begin{figure}

\centering{

\includegraphics[width=0.9\linewidth,height=\textheight,keepaspectratio]{fig_field_approach.png}

}

\caption{\label{fig-field}Treatment effect field showing heterogeneity
and structural boundary at \(X_1 > 160\). Gray region indicates
positivity violation where comparison is structurally impossible.}

\end{figure}%

\textbf{Key Messages:}

\begin{enumerate}
\def\labelenumi{\arabic{enumi}.}
\tightlist
\item
  Traditional methods are not wrong---Cox and IPTW correctly estimate
  average effects
\item
  But averages hide structure---HR = 0.55 does not reveal
  \textasciitilde3.5-fold variation
\item
  Positivity violation is information---the gray zone reflects clinical
  logic
\item
  Field-based output is a map, not a number
\end{enumerate}

\section{Relation to Existing Heterogeneity
Research}\label{relation-to-existing-heterogeneity-research}

\subsection{Differentiation from CATE}\label{differentiation-from-cate}

\begin{longtable}[]{@{}
  >{\raggedright\arraybackslash}p{(\linewidth - 4\tabcolsep) * \real{0.1951}}
  >{\raggedright\arraybackslash}p{(\linewidth - 4\tabcolsep) * \real{0.5122}}
  >{\raggedright\arraybackslash}p{(\linewidth - 4\tabcolsep) * \real{0.2927}}@{}}
\toprule\noalign{}
\begin{minipage}[b]{\linewidth}\raggedright
Aspect
\end{minipage} & \begin{minipage}[b]{\linewidth}\raggedright
CATE / HTE Research
\end{minipage} & \begin{minipage}[b]{\linewidth}\raggedright
Epidynamix
\end{minipage} \\
\midrule\noalign{}
\endhead
\bottomrule\noalign{}
\endlastfoot
Core question & ``What is the effect for subgroup \(X=x\)?'' & ``What is
the structure of the risk field?'' \\
Output & \(\tau(x) = \mathbb{E}[Y(1) - Y(0) \mid X=x]\) & \(\Phi(s)\),
\(\nabla \Phi\), structural boundaries \\
Positivity violation & Estimation problem (trim, extrapolate) &
\textbf{Information} (map the boundary) \\
Counterfactuals & Required & Not required \\
Goal & Better effect estimation & Different question entirely \\
\end{longtable}

\textbf{Key distinction:} CATE still asks ``what is the effect?''---just
conditional on \(X\) (Wager and Athey 2018). Epidynamix asks ``what does
the risk landscape look like, and where can we not go?'' Crucially, CATE
treats positivity violations as missing data; Epidynamix reports
boundaries explicitly as part of the result.

\subsection{Connection to Structural Nested
Models}\label{connection-to-structural-nested-models}

The local effect \(\delta_A \Phi(x)\) shares conceptual ground with the
``blip function'' in Structural Nested Mean Models (Robins 1994), which
also models treatment effects as functions of patient state. The key
difference is interpretive: SNMM aims to estimate causal effects under
sequential exchangeability, while Epidynamix treats the effect surface
as a geometric object to be mapped.

\subsection{Novelty Claim}\label{novelty-claim}

We do not claim to invent new mathematics. Potential landscapes exist in
physics and systems biology (e.g., Waddington's epigenetic landscape
(Waddington 1957)). Concurrent work by Leizerman (Leizerman 2025)
develops a ``Unified Causal Field Theory'' using differential geometry
and fiber bundles---a more general mathematical formalization applicable
across domains. Our contribution is complementary: applying this
geometric lens specifically to clinical RWD, where positivity violations
are common, effects are entangled with state, and the average-effect
question may be structurally unanswerable.

This is a \textbf{reinterpretation}, not an invention---a new language
for an old problem.

\section{Discussion}\label{discussion}

The Epidynamix framework offers a complementary perspective to standard
causal inference (Rubin 1974; Hernán and Robins 2020). It does not
replace existing methods; rather, it clarifies their domain of validity.

\textbf{When ATE works:} If the risk field is smooth, low-curvature, and
well-connected (positivity holds everywhere), the ATE is an accurate
summary.

\textbf{When ATE fails:} If effects are highly heterogeneous, or if
structural constraints create disconnected regions, the ATE may be
misleading or undefined. The field approach provides richer output---a
map rather than a number.

\textbf{Practical implications:}

\begin{enumerate}
\def\labelenumi{\arabic{enumi}.}
\tightlist
\item
  Report effect heterogeneity alongside average effects
\item
  Visualize the treatment effect landscape when possible
\item
  Treat positivity violations as findings, not errors
\item
  Consider whether the causal question is answerable before estimating
\item
  Report HR field maps (\(\exp\{-\delta_A \Phi\}\)) alongside scalar
  summaries
\end{enumerate}

\textbf{Limitations:} The framework is conceptual; practical estimation
of \(\Phi(s)\) requires flexible models (e.g., GAMs, neural networks)
and careful validation. Extension to time-varying treatments and
high-dimensional states remains an open challenge.

\section{Conclusion}\label{conclusion}

Newtonian mechanics is not wrong---it is flat-space physics. Causal
inference is not false---it is a low-curvature approximation.

Real-world epidemiology, with its structural constraints and
heterogeneous effects, may require a geometric view. The Epidynamix
framework offers one such lens: treating causal effects not as forces,
but as directional movements within a structured risk field.

When the standard causal question cannot be answered, perhaps we should
ask a different question.

\section{References}\label{references}

\phantomsection\label{refs}
\begin{CSLReferences}{1}{0}
\bibitem[\citeproctext]{ref-cole2008}
Cole, Stephen R, and Miguel A Hernán. 2008. {``Constructing Inverse
Probability Weights for Marginal Structural Models.''} \emph{American
Journal of Epidemiology} 168 (6): 656--64.

\bibitem[\citeproctext]{ref-hernan2020}
Hernán, Miguel A, and James M Robins. 2020. \emph{Causal Inference: What
If}. Boca Raton: Chapman \& Hall/CRC.

\bibitem[\citeproctext]{ref-leizerman2025}
Leizerman, Samuel L. 2025. {``Unified Causal Field Theory: A Proof of
Geometric Subsumption and Extension of Causal Inference Methods into
Unified Framework.''} SocArXiv.
\url{https://doi.org/10.31235/osf.io/c7pz9}.

\bibitem[\citeproctext]{ref-petersen2012}
Petersen, Maya L, Kristin E Porter, Susan Gruber, Yue Wang, and Mark J
van der Laan. 2012. {``Diagnosing and Responding to Violations in the
Positivity Assumption.''} \emph{Statistical Methods in Medical Research}
21 (1): 31--54. \url{https://doi.org/10.1177/0962280210386207}.

\bibitem[\citeproctext]{ref-robins1994}
Robins, James M. 1994. {``Correcting for Non-Compliance in Randomized
Trials Using Structural Nested Mean Models.''} \emph{Communications in
Statistics-Theory and Methods} 23 (8): 2379--2412.

\bibitem[\citeproctext]{ref-rubin1974}
Rubin, Donald B. 1974. {``Estimating Causal Effects of Treatments in
Randomized and Nonrandomized Studies.''} \emph{Journal of Educational
Psychology} 66 (5): 688.

\bibitem[\citeproctext]{ref-waddington1957}
Waddington, Conrad Hal. 1957. \emph{The Strategy of the Genes: A
Discussion of Some Aspects of Theoretical Biology}. London: George Allen
\& Unwin.

\bibitem[\citeproctext]{ref-wager2018}
Wager, Stefan, and Susan Athey. 2018. {``Estimation and Inference of
Heterogeneous Treatment Effects Using Random Forests.''} \emph{Journal
of the American Statistical Association} 113 (523): 1228--42.
\url{https://doi.org/10.1080/01621459.2017.1319839}.

\end{CSLReferences}




\end{document}
